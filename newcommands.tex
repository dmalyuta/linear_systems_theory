\newcommand*{\QED}{\ensuremath{\blacksquare}}
\newcommand{\union}{\cup}
\newcommand{\intersection}{\cap}
\newcommand{\overeq}[1]{\stackrel{\mathclap{\tiny{\substack{#1}}}}{=}}
\newcommand{\overto}[1]{\stackrel{\mathclap{\tiny{\substack{#1}}}}{\to}}
\newcommand{\Adj}{\textsc{Adj}}
\newcommand{\linA}{\mathcal{A}}
\newcommand{\linB}{\mathcal{B}}
\newcommand{\linC}{\mathcal{C}}
\newcommand{\inv}{^{-1}}
\newcommand{\perpoplus}{\overset{\perp}{\oplus}}
\newcommand{\e}[1]{e^{#1}}
\newcommand{\inver}[1]{#1^{-1}}
\newcommand{\fnorm}[2]{\|#1\|_{#2}}
\newcommand{\Property}[1]{
	\begin{properties}
		\item[\textit{Property #1}.]
	\end{properties}
}
\newcommand{\skewmat}[1]{\bm{#1}^{\times}}
\newcommand{\proj}[2]{\text{proj}_{#1}(#2)}
\newcommand{\tabitem}{~~\llap{\textbullet}~~}
\newcommand{\basis}[3]{\{#1_{#2}\}_{#2=1}^{#3}}
\newcommand{\sequence}[3]{\{#1_{#2}\}_{#2=0}^{#3}}
\newcommand{\Naturals}{\mathbb N}
\newcommand{\Reals}{\mathbb R}
\newcommand{\Complex}{\mathbb C}
\newcommand{\argmin}{\operatornamewithlimits{argmin}}
\newcommand{\argmax}{\operatornamewithlimits{argmax}}
\newcommand{\then}{\Rightarrow}
\newcommand{\Rank}{\textsc{Rank}}
\newcommand{\Range}{\textsc{Range}}
\newcommand{\Nullity}{\textsc{Nullity}}
\newcommand{\Null}{\textsc{Null}}
\newcommand{\Det}{\textsc{Det}}
\newcommand{\Spec}{\textsc{Spec}}
\newcommand{\DIM}{\text{dim}}
\newcommand*\bigcdot{\mathpalette\bigcdot@{.5}}
\newcommand{\InnerProd}[2]{\langle#1,#2\rangle}
\newcommand{\scaleval}{0.8}
\newcommand{\Expectation}[1]{\mathbb{E}\left[#1\right]}
\newcommand{\Variance}[1]{\text{Var}[#1]}
\newcommand{\Cov}[2]{\text{Cov}[#1,#2]}
\newcommand{\Prob}{\mathbb P}
\newcommand{\ord}[1]{^{\left(#1\right)}}
\DeclarePairedDelimiter\ceil{\lceil}{\rceil}
\DeclarePairedDelimiter\floor{\lfloor}{\rfloor}
\newcommand{\mc}[2]{{\color{#1}#2}}

%%%%%%%%%% Drawings
\newcommand*\circled[1]{\tikz[baseline=(char.base)]{\node[shape=circle,draw,inner sep=0pt,minimum size=0.1em] (char) {#1};}}

\newcommand*\circledcolor[2]{\tikz[baseline=(char.base)]{\node[shape=circle,draw,inner sep=0pt,minimum size=0em,fill=#2] (char) {#1};}}

\newcommand*\squaredcolor[2]{\tikz[baseline=(char.base)]{\node[shape=rectangle,draw,inner sep=0.5pt,minimum size=0.1em,fill=#2] (char) {#1};}}

\newcommand*\squaredcolornobord[2]{\tikz[baseline=(char.base)]{\node[shape=rectangle,inner sep=0.5pt,minimum size=0.1em,fill=#2] (char) {#1};}}

\newcommand\dangersign[1][1.5em]{%
  \renewcommand\stacktype{L}%
  \scaleto{\stackon[0.7pt]{\color{red}$\triangle$}{\tiny !}}{#1}%
}

%%%%%%%%%% Big commands
\newcommand{\distribdiscript}[6]{
\noindent\rule[0ex]{\linewidth}{0.5pt}
\flushleft{\textbf{#1 ($X\sim #2$)}} % Name (#1) and notation (#2)
\vspace{-2mm}
\noindent\rule[1.5ex]{\linewidth}{0.5pt}
#3 % Description of distribution
\begin{tabularx}{1\columnwidth}{X|X}
\hline
#4 % PMF/PDF/CDF go here
&
Properties:
#5 % Properties of distribution go here
\\ \hline
\end{tabularx}
\myfigure{
\includegraphics[width=0.9\columnwidth]{#6.pdf} % Filename of the distribution grpahic goes here
}
}

\newcommand{\nextchapter}[1]{
\begin{Chapter}
	\addtocounter{section}{1}
	\textbf{\thesection: #1}
\end{Chapter}
}

\newcommand{\nextsubchapter}[1]{
\begin{Subchapter}
	\textbf{#1}
\end{Subchapter}
}

\newcommand{\myfigure}[1]{
\vspace{-4mm}
\begin{figure}[H]
\centering
#1
\end{figure}
\vspace{-4mm}
}